\documentclass[a4paper,11pt]{article}

\usepackage[portuguese]{babel}
\usepackage[utf8]{inputenc}
\usepackage{indentfirst}
\usepackage{graphicx}
\usepackage{verbatim}
\usepackage{listings}
\usepackage{float}
\usepackage{hyperref}
\usepackage{dirtree}
\hypersetup{
  colorlinks=true,
  linkcolor=blue!70!red
}
\usepackage{color}
\usepackage{xcolor}
\usepackage[T1]{fontenc}

\begin{document}

\title{\Huge\textbf{Evacuações com Agentes BDI}\linebreak\linebreak\linebreak
\Large\textbf{Relatório Intercalar}\linebreak\linebreak
\linebreak\linebreak
\includegraphics[scale=0.1]{feup-logo.png}\linebreak\linebreak
\linebreak\linebreak
\Large{Mestrado Integrado em Engenharia Informática e Computação} \linebreak\linebreak
\Large{Agentes e Inteligência Artificial Distribuída}\linebreak\linebreak
\Large{Meter Aqui GRUPO}}
\author{
João Neto \\ 201203873 \\
\and
João Correia\\ 201101753 \\
\and
Miguel Mendes\\ 201105535 \\}
\date{\today}

\maketitle

\newpage

\tableofcontents

\newpage
\section{Introdução}
No âmbito da unidade curricular de Agentes e Inteligência Artificial Distribuída, do 4º ano do Mestrado Integrado de Engenharia Informática e Computação, este relatório tem o propósito de expor a abordagem que será adotada no projeto de Evacuações com Agentes BDI.

No sentido em que usará uma arquitetura de agentes baseada em \textit{Belief, Desire, Intention}, para estruturar todo um conjunto de indivíduos (agentes) que atuarão num contexto de simulação de uma evacuação. O programa permitirá a a influência do utilizador na manipulação do ambiente e irá tratar/simular possíveis situações de um acidente em que os agentes presentes no espaço se vão comportar de acordo com perfis implementados, que serão explicados mais a baixo.

\section{Objetivos}
O propósito deste projeto será simular a interação de agentes confinados a um espaço concreto e limitado. Este espaço será variável a nível de estrutura, podendo o utilizador desenhar os obstáculos e definir a posição dos agentes. Na descrição do cenário será feita uma abordagem mais aprofundada.

A simulação das ações dos agentes serão a base do projeto, sendo que o objetivo será implementar determinados perfis, através da alteração de \textit{plans} e \textit{tasks} dos agentes.

Com este espaço, o despoletar de eventos de acidentes e diversificando o comportamento dos indivíduos, o objetivo é recolher dados como tempo mínimo, médio e máximo de evacuação, perceber como é que a quantidade e tipo de agente influência a evacuação, número de feridos, entre outros.

\section{Método de Avaliação}
No programa, o tipo de perfil particular de cada agente influência como é que a interação é feita com os outros agentes.
Com isto é possível criar simulações com o propósito de avaliar como é que o tipo e quantidade de agentes do mesmo tipo influenciam toda uma mecânica de evacuação de um edifício.

O tempo é outro fator de importância pois vai permitir perceber como é que a disposição de saídas de emergência tem impacto numa evacuação, assim como o comportamento dos agentes presentes.

\section{Resultados Esperados}
Com as variáveis de:
\begin{itemize}
\item Agentes organizados, suficientes (este fator é de elevada importância) saídas de emergência e correta disposição das mesmas, que se irá também avaliar no projeto, espera-se 'ótimos' resultados quer a nível temporal e número mínimo/nulo de vítimas.

\item A variação de perfil e quantidade de cada agente irá contribuir tanto para resultados 'ótimos' como grandes desastres (Na parte dos agentes será referido, mas por exemplo a existência de Agentes cujo comportamento pode por em risco ou melhorar a situação de outros agentes irá causar grande influência.)

Assim, é impossível determinar o tipo de \textit{outcome} que terão as várias simulações efetuadas, pois uma pequena oscilação de agentes e espaços pode causar um impacto muito grande.
\end{itemize}

\section{Descrição do Cenário}

A criação do cenário terá como base um plano de duas dimensões, onde se existirão objetos colidíveis, sendo que o indivíduo terá que contornar ou evitar, objetos que projetarão outro objetos para cenário, no caso de simulação de um incêndio as chamas terão que se espalhar ou num desabamento pontos que ficarão inacessíveis. 


\section{Descrição dos Agentes}
Nesta fase do projeto estima-se a criação de 4 agentes principais:

Agente que ajuda agentes feridos:
\begin{itemize}
\item Agente cuja prioridade é alcançar a saída, mas se durante esse percurso se deparar com um agente ferido transporta-o com ele até à saída.
\\
\end{itemize}

Agente que não se importa com nada:
\begin{itemize}
\item Agente que só tem um e um só objetivo: sair. 
\item Qualquer tipo de obstáculo (ex: agente ferido) será ignorado, contornado ou ultrapassado. 
\\
\end{itemize}

Agente que 'congela':
\begin{itemize}
\item Agente cujo propósito será simular stress numa situação real. 
\item Quando o agente avista um agente ferido ou o ponto de início do acidente, entra em pânico. 
\\
\end{itemize}

Agente que guia:
\begin{itemize}
\item Ora para a criação de um agente que 'congela', é necessário alguém que o ajude.
\item Este agente ao contrário do primeiro agente referido, só atua em agentes 'congelados' e o seu papel é guia-los em fila para a saída.
\item O agente poderá adotar duas variantes, uma em que se desloca para a saída e guia agentes outrora 'congelados' ou uma em que procura agente que possam encontrar-se 'congelados' e só depois é que se desloca para a saída.
\end{itemize}

\section{Protocolos de Interação}

\section{Plataforma e Ferramenta}  
A ferramenta utilizada para implementar o projeto é o Jadex, utilizando o módulo bdiv3, que permite a definição de agentes em Java em vez do xml utilizado na anterior versão.
A estrutura dos projetos em bdiv3 é descrita em seguida:

\DTsetlength{.2em}{1.5em}{.4em}{.4pt}{3pt}

\dirtree{%
.1 Pasta Projeto.
.2 Application.xml.
.2 Pasta Agente.
.3 AgentBDI.java.
.3 Plan\_1.java.
.3 ....
.3 Task\_1.java.
.3 ....
.3 Service\_1.java.
.3 ....
.2 Pasta Capability.
.3 Capability.java.
.3 Plan\_1.java.
.3 ....
.3 Task\_1.java.
.3 ....
}


\end{document}