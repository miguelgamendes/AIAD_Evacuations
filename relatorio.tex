\documentclass[a4paper,11pt]{article}

\usepackage[portuguese]{babel}
\usepackage[utf8]{inputenc}
\usepackage{indentfirst}
\usepackage{graphicx}
\usepackage{verbatim}
\usepackage{listings}
\usepackage{float}
\usepackage{hyperref}
\hypersetup{
  colorlinks=true,
  linkcolor=blue!70!red
}
\usepackage{color}
\usepackage{xcolor}
\usepackage[T1]{fontenc}

\begin{document}

\title{\Huge\textbf{Evacuações com Agentes BDI}\linebreak\linebreak\linebreak
\Large\textbf{Relatório Intermédio}\linebreak\linebreak
\linebreak\linebreak
\includegraphics[scale=0.1]{feup-logo.png}\linebreak\linebreak
\linebreak\linebreak
\Large{Mestrado Integrado em Engenharia Informática e Computação} \linebreak\linebreak
\Large{Agentes e Inteligência Artificial Distribuída}\linebreak\linebreak
\Large{Meter Aqui GRUPO}}
\author{
João Neto \\ 201203873 \\
\and
João Correia\\ 201101753 \\
\and
Miguel Mendes\\ 201105535 \\}
\date{\today}

\maketitle

\newpage

\tableofcontents

\newpage
\section{Introdução}
No âmbito da unidade curricular de Agentes e Inteligência Artificial Distribuída, do 4º ano do Mestrado Integrado de Engenharia Informática e Computação, este relatório intermédio tem o propósito de expor a abordagem mais concreta que será adotada no projeto de Evacuações com Agentes BDI.

\section{Objetivos}
O propósito deste projeto será simular a interação de agentes confinados a um espaço concreto e limitado. Este espaço será variável a nível de estrutura mas não de dimensão. Na descrição do cenário será feita uma abordagem mais aprofundada.

A simulação das ações dos agentes serão a base do projeto, sendo que servirão .....

\section{Método de Avaliação}
No decorrer da simulação serão recolhidos os dados sobre o número de vítimas que não conseguiram escapar, o porquê e o tipo de agente. O propósito desta avaliação é perceber como é que o tipo de agente e o número de agentes de um certo tipo influenciam o resultado final de uma evacuação. 

PRECISO EXPOR MAIS INFORMAÇAO E EDITAR ESTA

\section{Resultados Esperados}
BLA BLA BLA


\section{Descrição do Cenário}

\section{Descrição dos Agentes}
Nesta fase do projeto estima-se a criação de 4 agentes principais:

JA AQUI OS METO XD

HELPING AGENT + DESCRIÇÃO

GRUMPY AGENT + DESCRIÇÃO

FRAGIL AGENT + DESCRIÇÃO

JESUS AGENT + DESCRIÇÃO


\section{Protocolos de Interação}

\section{Plataforma e Ferramenta}  

\end{document}